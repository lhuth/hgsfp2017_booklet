\subsection*{Fundamental physics with cold molecules}
\subsection*{S. Hoekstra, Uni Groningen}
\noindent Atomic and molecular physics has in recent years seen an strong evolution on two experimental fronts: the exquisite control over atomic and molecular samples, and the availability of extremely stable laser light over a wide frequency range. By combining these experimental precision tools with theory that explains atomic and molecular structure at a similar level of precision, a powerful field has emerged, providing a new perspective on the fundamental interactions and symmetries of our universe. Of special interest are the developments on cooling of molecules, which through huge enhancement effects are promising superior sensitivity. By combining precision table-top experiments, quantum structure calculations and particle-physics theory, the understanding of the foundations of our universe can be tested at energies that effectively surpass those available at the largest particle accelerators that have been built. In these lectures, I will give an overview of the main concepts and techniques in the field of cold molecules, especially where they are used to search for new physics.
