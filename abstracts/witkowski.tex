\subsection*{Folk theorems of Quantum Gravity}
\subsection*{L. Witkowski}
\noindent Is there anything substantial that we can say about a quantum theory
of gravity without immersing ourselves in the depths of String Theory and Loop
Quantum Gravity? There is indeed hope that one may get a glimpse at what
consistent quantum theories including gravity might look like. By performing
Gedankenexperimente with charged particles and black holes one may unearth
inconsistencies which can be cured if one imposes new physical theorems. These
are known as “folk theorems” of quantum gravity, the attribute “folk”
indicating that they do not come from a rigorous calculation. The beauty of
these folk theorems is that they are surprisingly simple statements, which
nevertheless lead to powerful constraints on low energy physics. For example,
the most famous folk theorem states that theories of quantum gravity do not
permit any global symmetries, which — if true — has far-reaching implications
for particle physics and cosmology. In my lectures I want to present some of
the best-known folk theorems and explain why we think they may be
true. Finally, I will review how these folk theorems act as guidance for
string theorists in their quest for understanding quantum gravity. 
