\subsection*{Single-Molecule Fluorescence and Super-Resolution Imaging}
\subsection*{K. Gru\ss mayer, Ecole Polytechnique Fédérale de Lausanne}
\noindent 
“A biological system can be exceedingly small. Many of the cells are very
tiny, but they are very active; they manufacture various substances; they walk
around; they wiggle; and they do all kinds of marvelous things – all on a very
small scale.” R. P. Feynman (1959)\newline

Light microscopy is one of the most important methods to investigate the
structure and function of (living) biological systems. These are ultimately
based on the dynamics and interactions of individual molecules. In the last
decades, developments in experimental physics along with advances in
(bio-)chemistry were the driving force that allowed investigating those
processes at the molecular level. I will concentrate on fluorescence imaging,
which is ideally suited to study single-molecules due to its sensitivity. I
will introduce important milestones in this young field, along with the major
microscopy techniques, photophysical processes and statistical analysis
required to obtain quantitative information.  In the second part of the
lecture, I will focus on how the properties of fluorophores along with a
clever optical design can be used to overcome Abbe’s diffraction limit of
light microscopy. For their “development of far-field super-resolved
microscopy”, three physicists were awarded the Nobel Prize in Chemistry
2014. I will discuss how the techniques can be pushed to reach the limits – a
spatial resolution approaching the size of single-molecules. 
