\subsection*{Environmental Geophysics using Electromagnetic Methods}
\subsection*{P. Yogeshwar, University of  Cologne}
\noindent
Geoelectrical and electromagnetic geophysical techniques have become very popular for various environmental applications such as for example groundwater exploration/management, investigation of waste disposal sites and detection of suitable landfills. These surface geophysical methods are non-invasive and provide valuable information of the subsurface physical properties. The resolved physical parameter (e.g rock density, magnetic susceptibility and electrical conductivity) depends on the applied geophysical exploration technique. Geoelectrical and electromagnetic methods are sensitive to the subsurface electrical conductivity. The electrical conductivity spans approximately 25 decades of magnitude for materials occurring in the earth's subsurface. Massive dry rocks exhibit extremely low electrical conductivities and are nearly transparent for electromagnetic methods, whereas for example a contaminated aquifer due to saline water intrusion serves as a suitable target. Due to the close relationship between electrical conductivity and (some) hydrogeological properties of aquifers, geoelectrical and electromagnetic techniques are the most popular in groundwater exploration.\\
The first part of this lecture gives a brief overview of geoelectrical and electromagnetic geophysical methods with respect to their typical areas of application, advantages/disadvantages and the different subsurface electrical conductivity mechanisms. The second part focuses comprehensively on the transient electromagnetic method (TEM) using mainly inductively ground-coupled loop sources. The TEM method has gained extreme popularity for environmental targets in the past decades and is commonly used for groundwater exploration. The lecture discusses various important aspects with respect to the TEM method, namely: basic methodological concepts, common measurement devices, typical TEM survey layouts, general noise-sources and coupling effects, processing techniques, interesting TEM source-field modifications and, finally, the conventional forward and inverse modeling techniques to reconstruct the subsurface electrical conductivity distribution. In the third part, recent real-world case studies are comprehensively discussed. A focus lies on an exemplary study from the Azraq Basin area. This area is of enormous economical importance to Jordan. Approximately one third of the freshwater for Jordan’s capital city of Amman is provided from the Azraq area. The extensive groundwater exploitation has led to a severe decline in the groundwater table. In the central part of the area, the groundwater is hyper-saline. To ensure the freshwater supply, groundwater research has been (and still is) an ongoing and relevant issue over the past 30 years. The presented geophysical investigations provide information about the extent of the saline water body and aim to support the ongoing groundwater management in the area.
